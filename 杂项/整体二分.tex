给你n个数,多次询问某段区间第k小的数。$(1 \le n \le 100000, 1 \le m \le 5000)$

所谓整体二分,需要数据结构题满足以下性质:

1.询问的答案具有可二分性。

2.修改对判定答案的贡献相对独立,修改之间互不影响效果。

3.修改如果对判定答案有贡献,则贡献为一确定的与判定标准无关的值。

4.贡献满足交换律,结合律,具有可加性。

5.题目允许离线操作。

询问的答案有可二分性质显然是前提,我们发现,因为修改对判定标准的贡献相对独立,且贡献的值(如果有的话)与判定标准无关,所以如果我们已经计算过某一个些修改对询问的贡献,那么这个贡献永远不会改变,我们没有必要当判定标准改变时再次计算这部分修改的贡献,只要记录下当前的总贡献,再进一步二分时,直接加上新的贡献即可。

这样的话,我们发现,处理的复杂度可以不再与序列总长度直接相关了,而可能只与当前待处理序列的长度相关。


接下来考虑本题。

对于单个查询而言,我们可以采用预处理+二分答案的方法解决,但现在我们要回答的是一系列的查询,对于查询而言我们都要重新预处理然后二分,时间复杂度无法承受,但是我们仍然希望通过二分答案的思想来解决,整体二分就是基于这样一种想法,我们将所有操作(包括修改和查询)一起二分,进行分治。

我们时刻维护一个操作序列和对应的可能答案区间 [L,R],我们先求的一个判定答案 mid=(L+R)>>1,然后我们考虑操作序列的修改操作,将其中符合条件的修改对各个询问的贡献统计出来,然后我们对操作序列进行划分。