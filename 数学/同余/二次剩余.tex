一个数 $a$ ,如果不是 $p$ 的倍数且模 $p$ 同余于某个数的平方,则称 $a$ 为模 $p$ 的 二次剩余 。而一个不是 $p$ 的倍数的数 $b$ ,不同余于任何数的平方,则称 $b$ 为模 $p$ 的非二次剩余 。

对二次剩余求解,也就是对常数 $a$ 解下面的这个方程:

$$
x^2 \equiv a \pmod p
$$

通俗一些,可以认为是求模意义下的开方。这里只讨论 $\boldsymbol{p}$  为奇素数的求解方法,将会使用 Cipolla 算法。

解的数量:

对于 $x^2 \equiv n \pmod p$ ,能满足" $n$ 是模 $p$ 的二次剩余"的 $n$ 一共有 $\frac{p-1}{2}$ 个(0 不包括在内),非二次剩余有 $\frac{p-1}{2}$ 个。

Cipolla 算法:

找到一个数 $a$ 满足 $a^2-n$ 是非二次剩余,至于为什么要找满足非二次剩余的数,在下文会给出解释。
这里通过生成随机数再检验的方法来实现,由于非二次剩余的数量为 $\frac{p-1}{2}$ ,接近 $\frac{p}{2}$ ,所以期望约 2 次就可以找到这个数。

建立一个"复数域",并不是实际意义上的复数域,而是根据复数域的概念建立的一个类似的域。
在复数中 $i^2=-1$ ,这里定义 $i^2=a^2-n$ ,于是就可以将所有的数表达为 $A+Bi$ 的形式,这里的 $A$ 和 $B$ 都是模意义下的数,类似复数中的实部和虚部。

在有了 $i$ 和 $a$ 后可以直接得到答案, $x^2\equiv n\pmod p$ 的解为 $(a+i)^{\frac{p+1}{2}}$ 。
