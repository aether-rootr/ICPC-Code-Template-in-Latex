中国剩余定理 (Chinese Remainder Theorem, CRT) 可求解如下形式的一元线性同余方程组(其中 $n_1, n_2, \cdots, n_k$ 两两互质):

$$
\begin{cases}
x &\equiv a_1 \pmod {n_1} \\
x &\equiv a_2 \pmod {n_2} \\
  &\vdots \\
x &\equiv a_k \pmod {n_k} \\
\end{cases}
$$

算法流程:

1. 计算所有模数的积 $n$ ;
2.  对于第 $i$ 个方程:
    1. 计算 $m_i=\frac{n}{n_i}$ ;
    2. 计算 $m_i$ 在模 $n_i$ 意义下的逆元  $m_i^{-1}$ ;
    3. 计算 $c_i=m_im_i^{-1}$ ( 不要对 $n_i$ 取模 )。
3. 方程组的唯一解为: $a=\sum_{i=1}^k a_ic_i \pmod n$ 。

应用:

某些计数问题或数论问题出于加长代码、增加难度、或者是一些其他不可告人的原因,给出的模数:不是质数!

但是对其质因数分解会发现它没有平方因子,也就是该模数是由一些不重复的质数相乘得到。

那么我们可以分别对这些模数进行计算,最后用 CRT 合并答案。

下面这道题就是一个不错的例子。

    给出 $G,n$ ( $1 \leq G,n \leq 10^9$ ),求:
    
    $$
    G^{\sum_{k\mid n}\binom{n}{k}} \bmod 999~911~659
    $$

首先,当 $G=999~911~659$ 时,所求显然为 $0$ 。

否则,根据 欧拉定理 ,可知所求为:

$$
G^{\sum_{k\mid n}\binom{n}{k} \bmod 999~911~658} \bmod 999~911~659
$$

现在考虑如何计算:

$$
\sum_{k\mid n}\binom{n}{k} \bmod 999~911~658
$$

因为 $999~911~658$ 不是质数,无法保证 $\forall x \in [1,999~911~657]$ , $x$ 都有逆元存在,上面这个式子我们无法直接计算。

注意到 $999~911~658=2 \times 3 \times 4679 \times 35617$ ,其中每个质因子的最高次数均为一,我们可以考虑分别求出 $\sum_{k\mid n}\binom{n}{k}$ 在模 $2$ , $3$ , $4679$ , $35617$ 这几个质数下的结果,最后用中国剩余定理来合并答案。

也就是说,我们实际上要求下面一个线性方程组的解:

$$
\begin{cases}
x \equiv a_1 \pmod 2\\
x \equiv a_2 \pmod 3\\
x \equiv a_3 \pmod {4679}\\
x \equiv a_4 \pmod {35617}
\end{cases}
$$

而计算一个组合数对较小的质数取模后的结果,可以利用卢卡斯定理。

比较两 CRT 下整数:

考虑 CRT, 不妨假设 $n_1\leq n_2 \leq ... \leq n_k$ 

$$
\begin{cases}
x &\equiv a_1 \pmod {n_1} \\
x &\equiv a_2 \pmod {n_2} \\
  &\vdots \\
x &\equiv a_k \pmod {n_k} \\
\end{cases}
$$

与 PMR(Primorial Mixed Radix) 表示

 $x=b_1+b_2n_1+b_3n_1n_2...+b_kn_1n_2...n_{k-1} ,b_i\in [0,n_i)$ 

将数字转化到 PMR 下,逐位比较即可

转化方法考虑依次对 PMR 取模

$$
\begin{aligned}
b_1&=a_1 \bmod n_1\\
b_2&=(a_2-b_1)c_{1,2} \bmod n_2\\
b_3&=((a_3-b_1')c_{1,3}-x_2')c_{2,3} \bmod n_3\\
&...\\
b_k&=(...((a_k-b_1)c_{1,k}-b_2)c_{2,k})-...)c_{k-1,k} \bmod n_k
\end{aligned}
$$

其中 $c_{i,j}$ 表示 $n_i$ 对 $n_j$ 的逆元, $c_{i,j}n_i \equiv 1 \pmod {n_j}$ 

扩展:模数不互质的情况

两个方程:

设两个方程分别是 $x\equiv a_1 \pmod {m_1}$ 、 $x\equiv a_2 \pmod {m_2}$ ;

将它们转化为不定方程: $x=m_1p+a_1=m_2q+a_2$ ,其中 $p, q$ 是整数,则有 $m_1p-m_2q=a_2-a_1$ 。

由裴蜀定理,当 $a_2-a_1$ 不能被 $\gcd(m_1,m_2)$ 整除时,无解;

其他情况下,可以通过扩展欧几里得算法解出来一组可行解 $(p, q)$ ;

则原来的两方程组成的模方程组的解为 $x\equiv b\pmod M$ ,其中 $b=m_1p+a_1$ , $M=\text{lcm}(m_1, m_2)$ 。

多个方程:

用上面的方法两两合并就可以了……